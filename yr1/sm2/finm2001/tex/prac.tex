\documentclass{report}
\setlength{\parindent}{0pt}
\begin{document}
\section{WACC}
\label{sec:wacc}
 Weighted Average Cost of Capital\\

\textbf{Pre-Tax WACC:} $\frac{E}{E+R}r_e+\frac{D}{E+D}r_d$

\textbf{After-Tax WACC:} $\frac{E}{E+R}r_e+\frac{D}{E+D}r_d*(1-\tau)$ 

\section{Investment Decisions}
\label{sec:invdec}


\subsection{NPV}
\label{subsec:npv}
Net present value. If $>$ 0, accept, 0 $=<$ decline

\vspace*{1\baselineskip}

\textbf{NPV:} $(1-\tau)(R-E)-\tau D_t-I_t$

\subsection{IRR}
\label{sec:irr}

\textbf{IRR} is the \textit{Internal Rate of Return}. It is the discount rate that causes a project's NPV to equal 0. Can be used to compare shit i think???

\textit{Does not apply when dealing with delayed investmens, a nonexistent IRR or multiple IRRs}

\vspace*{1\baselineskip}
\textbf{IRR:} find IRR such that $0=\sum_{t=1}^n\frac{C_t}{(1+IRR)^t}-C_0$

\section{Capital Budgeting}
\label{sec:capital-budgeting}

Methods of comparing projects with different lives:
\begin{itemize}
\item \textbf{Equivalent Annual Value method:} $\frac{NPV}{[\frac{1-(1+r)^{-n}}{r}]}$
\item Constant chain of replacement in perpetuity method (idk what this is)
\item lowest common multiple method (idk what this is either so LUL)
\end{itemize}

\textbf{Earnings Before Interest and Tax:} EBIT = Sales - COGS - Genreal Expenses - R\&D - Depreciation.

EBIT - Income Tax = Unlevered Net Income

\textbf{Cannibalization:} when sales of a new product displace sales of an existing product. This \textit{is} included in the decision making process, as is \textbf{rent}\pagebreak

\section{Leasing}
\label{sec:leasing}

Types of leases:
\begin{description}
\item[Sales-type Lease:] the lessor is the manufacturer of the asset (e.g. a photocopier from Xerox)
\item[Direct Lease:] lessor is not the manufacturer, often an independent company that specialises in purchasing assets and leasing them to customers 
\item[Sales and Lease-back:] a firm already owns an asset it would prefer to lease. Firm sells asset, gets cash, leases back from the new owner and makes lease payments. \textit{Useful for firms facing a funding short fall or firms that are asset rich but cash poor.}
\item[Leveraged Lease:] Lessor borrows from a bank or other lender to obtain the initial capital to purchase an asset, using the lease payments to pay back the bank loan  
\end{description}
\end{document}

%%% Local Variables:
%%% mode: latex
%%% TeX-master: t
%%% End:
