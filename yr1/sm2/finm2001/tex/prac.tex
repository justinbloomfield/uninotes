\documentclass{report}
\setlength{\parindent}{0pt}
\begin{document}
\section{Investment Decisions}
\label{sec:invdec}


\subsection{NPV}
\label{subsec:npv}
Net present value. If $>$ 0, accept, 0 $=<$ decline

\vspace*{1\baselineskip}

\textbf{NPV:} $(1-\tau)(R-E)-\tau D_t-I_t$

\subsection{IRR}
\label{sec:irr}

\textbf{IRR} is the \textit{Internal Rate of Return}. It is the discount rate that causes a project's NPV to equal 0. Can be used to compare shit i think???

\textit{Does not apply when dealing with delayed investmens, a nonexistent IRR or multiple IRRs}

\vspace*{1\baselineskip}
\textbf{IRR:} find IRR such that $0=\sum_{t=1}^n\frac{C_t}{(1+IRR)^t}-C_0$
\newpage
\section{Capital Budgeting}
\label{sec:capital-budgeting}

Methods of comparing projects with different lives:
\begin{itemize}
\item \textbf{Equivalent Annual Value method:} $\frac{NPV}{[\frac{1-(1+k)^{-n}}{k}]}$
\item Constant chain of replacement in perpetuity method. $NPV_{\infty}=NPV_0[\frac{(1+k)^n}{(1+k)^n-1}]$
  \textit{or}
  $NPV_{\infty}=\frac{\mbox{EAV}}{k}$

 
\item lowest common multiple method (idk what this is either so LUL)
\end{itemize}

\textbf{Earnings Before Interest and Tax:} EBIT = Sales - COGS - Genreal Expenses - R\&D - Depreciation.

EBIT - Income Tax = Unlevered Net Income


\vspace*{1\baselineskip}
\textbf{Profitability Index:} $\frac{\mbox{NPV}}{\mbox{Resources Consumed}}$

\vspace*{1\baselineskip}
\textbf{Cannibalization:} when sales of a new product displace sales of an existing product. This \textit{is} included in the decision making process, as is \textbf{rent}

\textbf{Net Working Capital (NWC):} Current Assets - Current Liabilities

\textbf{$\Delta NWC_t=NWC_t-NWC_{t-1}$}
\newpage
\section{Leasing}
\label{sec:leasing}

Types of leases:
\begin{description}
\item[Sales-type Lease:] the lessor is the manufacturer of the asset (e.g. a photocopier from Xerox)
\item[Direct Lease:] lessor is not the manufacturer, often an independent company that specialises in purchasing assets and leasing them to customers 
\item[Sales and Lease-back:] a firm already owns an asset it would prefer to lease. Firm sells asset, gets cash, leases back from the new owner and makes lease payments. \textit{Useful for firms facing a funding short fall or firms that are asset rich but cash poor.}
\item[Leveraged Lease:] Lessor borrows from a bank or other lender to obtain the initial capital to purchase an asset, using the lease payments to pay back the bank loan  
\end{description}

End-of-term lease options:
\begin{description}
\item[Fair Market Value (FMV) Lease:] gives the lessee the option to purchase the asset at it's fair market value at the termination of the lease
\item[\$1 Out Lease:] ownership is transferred to the lessee at the end of the lease for a nominal cost of \$1. The point of doing this over purchasing the asset outright is for tax reasons (explained further down)
\item[Fixed Price Lease:] the lessee has the option to purchase the asset at the end of the lease for a fixed price that is set upfront in the lease contract. This option is accompanied by a higher lease rate to compensate for the value of this option. The lesse should exercise when the P(market) $>$ P(fixed) and not exercise when P(market) $<$ P(fixed)
\item[Fair Market Value Cap Lease:] lessee can purchase the asset at the minimum of its fair market value and a fixed price or ``cap''  
\end{description}

Tax Treatment of Leases - 2 Types:
\begin{description}
\item[True Tax Lease (Operating Lease):] the \textit{lessor} receives the depreciation deduction associated with the ownership of the asset.
  \begin{itemize}
  \item the \textit{lessee} reports the entire lease payment as an operating expense. They do not deduct a depreciation expense for the asset and does not report the asset, or the lease payment liability, on its balance sheet. The lease is simply a rental agreement for the lessee
  \item the lessor receives the depreciation tax shield but pays tax on the lease payments
  \end{itemize}
\item[Non-Tax Lease (Financial or Capital Lease):] the \textit{lessee} receives the depreciation deductions for tax purposes and can also deduct the interest portion of the lease payments as an interest expense
  \begin{itemize}
  \item Viewed as an acquisition for accounting purposes; the lesses lists the asset on its balance sheet and incurs depreciation expenses that are tax deductible
  \item the lessee also lists the present value of the future lease payments as a liability and deducts the interest portion of the lease paymenst as an interest expense. \textbf{Only} the interest portion of the lease payments can be deducted. Otherwise, the depreciation deduction would be double counted.
  \item The interest portion of the lease payment is interest income to lessor, subject to tax liability
  \end{itemize}
\end{description}

\vspace*{1\baselineskip}
If the lease satisfies any of these six conditions, it's treated as a non-tax lease (reported on balance sheet, increasing the leverage of the firm):
\begin{itemize}
\item the lessee obtains equity in the leased asset (some kind of ownership in the leased asset)
\item the lessee receives ownership of the asset on completion of all lease payments (e.g. the \$1 out lease)
\item The total amount that the lessee is required to pay for a relatively short period of use constitutes an inordinately large proportion of the toal value of the asset. For example, if the asset has an economic life of 10 years and the firm leases it for 3 years with the payments covering 90\% of the asset's value, it is a non-tax lease
\item The lease payments greatly exceed the current fai rental value of the asset
\item The property may be acquired at a bargain price in relation to the fair market value of the asset at the time when the option may be exercised (e.g. \$1 out lease again)
\item some portion of the lease payments is specifically designated as interest or its equivalent (transaction is structured more like debt instead of rent)
\end{itemize}
\textit{\textbf{Remember that the first lease payment occurs at time 0!} Therefore use \emph{annuity-due formula}, $F[\frac{(1+r)^{n-1}}{r}](1+r)$}
\vspace*{1\baselineskip}

\textbf{Loan Balance:} $PV[\mbox{Future FCF of Lease Versus Buy at } r_D(1-\tau_c)]$

\vspace*{1\baselineskip}
$PV(\mbox{Lease Payments}) = \mbox{Purchase Price} - PV(\mbox{Residual Value})$

\section{Financing Choices}
\label{sec:financ-choic}

just theory shit, not super duper important, but know the main concepts

\section{Capital Structure}
\label{sec:capital-structure-1}

\section{Capital Structure with Taxes}
\label{sec:capit-struct-2}

 \section{Capital Structure (Financial Distress, Agency Costs and Trade-Off Theory)}
\label{sec:capit-struct-3}

$V^L=V^U+PV(\mbox{Interest Tax Shield})$

\textbf{Trade-Off Theory:} $V^L = V^U + PV(\mbox{Interest Tax Shield}) - PV(\mbox{Financial Distress Costs})$
 
\section{Payout Policy}
\label{sec:payout-policy}

Cum dividend: $P_{cum}=\mbox{Current Dividend} + PV(\mbox{Future Dividends})$

Effective dividend tax rate: $\tau_d^*=(\frac{\tau_d-\tau_g}{1-\tau_g})$

\newpage
\section{Cost of Capital and WACC}
\label{sec:cost-capital-wacc}

\textbf{Pre-Tax WACC\footnote{Also known as Asset CoC}:} $r_u=\frac{E}{E+R}r_e+\frac{D}{E+D}r_d$

\textbf{After-Tax WACC:} $r_{wacc}=\frac{E}{E+R}r_e+\frac{D}{E+D}r_d*(1-\tau)$

\vspace*{1\baselineskip}
To estimate $r_e$, use the CAPM: $E(R_i)=r_f+\beta_i[E(R_m)-r_f]$

\vspace*{1\baselineskip}
Net Debt = Debt - Excess Cash and Cash Equivalents (short-term investments)\footnote{ Firms with more cash than debt are negatively leveraged, giving a -D}

Cost of debt = $\frac{\mbox{interest expense}}{\mbox{interest-bearing debt}}$

Risk Premium = $E[R_{mkt}]-r_f$

\vspace*{1\baselineskip}
Total risk: measured by \textit{volatility}

Market risk: measured by \textit{beta}

\vspace*{1\baselineskip}

\subsection{WACC Method}
\label{sec:wacc-method}

\textbf{Using the WACC to value a project:} \[ V_0^L= \frac{FCF_1}{(1+r_{wacc})} + \frac{FCF_2}{(1+r_{wacc})^2} + ... + \frac{FCF_n}{(1+r_{wacc})^n} + L \]

Summary of the WACC method:
\begin{enumerate}
\item Determine the free cash flows of the investment
\item Compute the WACC, including tax benefit of leverage
\item Compute the value of the investment, including the tax benefit of leverage, by discounting the free cash flow of the investment using the after-tax WACC
\end{enumerate}

\subsection{APV Method}
\label{sec:apv-method}

Uses the \textbf{pre-tax WACC}

\[ APV = V^U + PV{(\mbox{Interest Tax Shield})}\]

Interest paid in year $t = r_d * D_{t-1}$

\vspace*{1\baselineskip}
Summary of the APV Method:
\begin{enumerate}
\item Determine the investment's value without leverage
\item Determine the present value of the interest tax shield
  \begin{enumerate}
  \item Determine the expected interest tax shield
  \item Discount the interest tax shield at the pre-tax WACC
  \end{enumerate}
\item Add the unlevered value to the presten value of the interest tax shield to determine the value of the investment with leverage \end{enumerate}

\section{Extensions of Capital Budgeting}
\label{sec:extens-capit-budg}

\end{document}

%%% Local Variables:
%%% mode: latex
%%% TeX-master: t
%%% End:
