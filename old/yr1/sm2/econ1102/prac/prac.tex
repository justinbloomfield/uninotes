\documentclass{article}
\setlength{\parindent}{0pt}
\begin{document}
\section{Unemployment}
\label{sec:unemp}

Types of unemployment and their meanings:
\begin{itemize}
\item seasonal: to do with time periods (e.g. actual seasons, elections, tourism) 
\item structural: \textbf{mismatch of skills and requirements of work}
\item frictional: short term, \textbf{stemming from process of matching workers with jobs, e.g. university finishers and school leavers} 
\item cyclical: to do with business cycle contraction/expansion \textit{(but mostly contraction)}
\end{itemize}

\section{National Income Accounting}
\label{sec:nati-income-acco}
\textbf{GDP}: market value of the final goods and services produced in an economy over a certain period

\subsection{Types}
\label{sec:types}
\begin{description}
\item [Income Measure:] measures the sum of all income earned
\item [Expenditure Measure:] counts the total purchases in the economy
\item [Production Measure:] counts the number of goods produced in the economy
\end{description}
\textit{Production = Expenditure = Income}

\subsection{Formulae}
\label{sec:formulae}
GDP = $C + I + G + NX$ (\textit{where} $NX$ = \textit{Exports - Imports})

\subsection{Notes}
\label{sec:notes}
Transfer payments aren't included, so shit like centrelink doesn't count because that would be double counting it

\section{Multipliers}
\label{sec:multipliers}

\begin{description}
\item [Multiplier Effect:] The process by which an increase in autonomous expenditure leads to a larger increase in real GDP
\item [Government Purchases Multiplier:] $\frac{\mbox{change in equilibrium real GDP}}{\mbox{change in government purchase}}$
\item [Tax Multiplier:] $\frac{\mbox{change in equilibrium real GDP}}{\mbox{change in taxes}}$
\end{description}

\section{AS/AD Model}
\label{sec:asad}
Real GDP and the price level are determined in the short run by the intersection of the AS/AD curves
\subsection{Aggregrate Demand}
\label{sec:ad}
\textit{Shows the relationship between the price level and the quantity of real GDP demanded by households, firms and the government (i.e. the whole economy).}\\


Slopes downwards because of:
\begin{itemize}
\item wealth effect
\item interest rate effect 
\item trade effect\\
\end{itemize}

Variables that shift the AD curve:
\begin{itemize}
\item changes in government policies
\item changes in expectations of households and firms
\item changes in foreign variables in outside economies\\
\end{itemize}
\subsection{Aggregate Supply}
\label{sec:as}
\textit{Shows the relationship in the between the price level and the quantity of real GDP supplied}
\begin{description}
\item[Long-run AS curve:] a curve that shows the relationship the the long run between the price level and the quantity of real GDP supplied. Shows that in the LR increases in the price level do not affect the level of real GDP
\item[Short-run AS curve:] a curve that shows (in the short run) firms will produce more in response to higher prices. This is because generally the prices of inputs tend to rise more slowly than the prices of the final products.\\
\end{description}

Changes in the price level are depicted in movements up or down the stationary SRAS curve. Exogenous shocks cause the SRAS curve to shift.\\

Variables that shift \textit{both} the SRAS and LRAS curves:
\begin{itemize}
\item increase in the labour force and/or in the capital stock and/or in resources
\item technological change 
\end{itemize}

\section{Aggregate Expenditure}
\label{sec:ae}
\textit{\textbf{Total amount of spending in the economy} - sum of consumption, planned investment, government purchases and net exports}
\section{Policies}
\label{sec:pol}

\subsection{Monetary Policy}
\label{sec:monpol}
\textit{Actions taken by central banks to manage interest rates in the pursuit of macroeconomic goals}
\begin{description}
\item[Quantity Theory of Money:] $M*V = P*Y$\\
  M = money supply; V = velocity of money; P = price level; Y = real GDP
\item[Velocity of Money:] $V = \frac{P*Y}{M}$
\item[Growth Rates:] $\Delta M+\Delta V=\Delta P+\Delta Y$\\ Probs a more useful form: c $\Delta P = \Delta M + \Delta V - \Delta Y$
  
\end{description}

Tries to influence market interest rate, liquidity in the economy and the money market equilibrium (i.e. dynamics of supply and demand for money)\\


Uses three "levers":
\begin{itemize}
\item \textbf{Cash rate:} the interest rate on loans in the overnight money market. Set directly by the RBA in Aus. 
\item \textbf{Open Market Operations (OMOs):} the central bank purchasing or selling financial securities such as bonds as way of indirectly setting the market interest rate, of which the cash rate is only part
\item \textbf{Money supply management}: modulating the money base and overall system liquidity and credit (fairly directly)
\end{itemize}
Changes in variables other than the interest rate cause the money demand curve to shift. The two most important of these are:
\begin{itemize}
\item \textbf{Real GDP:} The greater income is, the greater the demand for money
\item \textbf{Price level:} as (nominal) prices increase, you need more (nominal) cash to make purchases
\end{itemize}

The \textit{monetary supply curve} is a \underline{\textbf{vertical}} line if the RBA is monetary targeting, and a \underline{\textbf{horizontal}} line if the RBA is targeting the interest rate. 

\begin{description}
\item[Expansionary Monetary Policy ('loose'):] aims to decrease interest rates to increase GDP 
  \begin{itemize}
  \item done through OMO and cash rate decreases
  \item AD curve shifts further to the right
  \end{itemize}
  
\item[Contractionary Monetary Policy ('tight'):] aims to increase interest rates to decrease inflation
  \begin{itemize}
  \item also done through OMO and cash rate alterations (i would assume increases?)
  \item used during periods of high or rising inflation rates
  \item AD curve still shifts to the right, but less than it would have without the policy
  \end{itemize}
\end{description}
\subsection{Fiscal Policy}
\label{sec:fiscpol}
\textit{Think of fiscal policy as AD management, with spending and taxing to as policy instrumenst to act on $Y = C+I+G+NX$}

\begin{description}
\item[Expansionary Fiscal Policy:] involves increasing discretionary government purchases and lowering taxes in order to increase aggregate demand. Appropriate for when the economy is in a recession.
\item[Contractionary Fiscal Policy:] involves decreasing discretionary government purchases and/or increasing taxes in order to lessen aggregate dmenad. Apppropriate when the economy is above full employment equilibrium and  mthe inflation rate is too high
\end{description}

\section{Minsky}
\label{sec:minsk}
Minsky argues basically two things:
\begin{itemize}
\item that the financial system has an intricate interdependency with the real secor at all times
\item that the nature of this interdependency is shaped by innate characteristics of human psychology, especially crowd thinking and the peculiarities of human cognition and judgement\\
\end{itemize}
Other claims:
\begin{itemize}
\item financial booms and busts are inevitable and that this reality must be incorporated into macroeconomic policy
\end{itemize}
\end{document}
%%% Local Variables:
%%% mode: latex m
%%% TeX-master: t
%%% End: