% Created 2016-11-12 Sat 16:14
\documentclass[11pt]{article}
\usepackage[utf8]{inputenc}
\usepackage[T1]{fontenc}
\usepackage{fixltx2e}
\usepackage{graphicx}
\usepackage{longtable}
\usepackage{float}
\usepackage{wrapfig}
\usepackage{rotating}
\usepackage[normalem]{ulem}
\usepackage{amsmath}
\usepackage{textcomp}
\usepackage{marvosym}
\usepackage{wasysym}
\usepackage{amssymb}
\usepackage{hyperref}
\tolerance=1000
\author{Justin Bloomfield}
\date{\today}
\title{studynotes}
\hypersetup{
  pdfkeywords={},
  pdfsubject={},
  pdfcreator={Emacs 25.1.1 (Org mode 8.2.10)}}
\begin{document}

\maketitle
\section{Notes}
\label{sec-1}
\subsection{Calculus}
\label{sec-1-1}
\subsubsection{W8}
\label{sec-1-1-1}
\begin{enumerate}
\item Integration of Rational Functions, Section 9.5
\label{sec-1-1-1-1}
for shit in the form of $f(x)=\frac{g(x)}{h(x)}$
\begin{enumerate}
\item If the numerator is a higher degree function than the denominator, must use long division
\begin{itemize}
\item $\square$ Add example to cheat sheet
\end{itemize}
\item Use partial fractions to separate function
\begin{itemize}
\item Factorise denominator first, to split
\item do the other stuff you need to do (shouldn't need an example for this)
\end{itemize}
\item Integrate as normal
\end{enumerate}
\item Improper Integrals, Section 9.6
\label{sec-1-1-1-2}
An integral is \textbf{improper} if it is unbounded at an end point, or if an endpoint is infinite.
\begin{itemize}
\item $\square$ Add example to cheat sheet (be sure to include the limits)
\end{itemize}
\item Taylor Polynomials, Sections 10.1-10.2
\label{sec-1-1-1-3}
\textbf{Lagrange's Remainder Theorem}
$En(x)= \frac{f^{(n+1)}(x_{0})}{(n+1)!}(x-a)^{n+1}$
Used for determining the error in the approximation

\textbf{Taylor Polynomials}
$\sum^{n}_{i=0}\frac{f^{(i)}(a)}{i!}(x-a)^{i}$

\begin{itemize}
\item $\square$ Defs example for this one
\end{itemize}
\end{enumerate}
\subsubsection{W9}
\label{sec-1-1-2}
\begin{enumerate}
\item Taylor Series, Sections 10.3-10.5
\label{sec-1-1-2-1}
The \textbf{Taylor Series} of $f$ about $a$ is the infinite sum:
$\sum^{\infty}_{i=0}\frac{f^{(i)}(a)}{i!}(x-a)^{i}$ 

I don't have a fucking clue how to do these, so do make sure to 
\begin{itemize}
\item $\square$ include an example
\end{itemize}
\end{enumerate}
\subsubsection{W10}
\label{sec-1-1-3}
\begin{enumerate}
\item Differential Equations, Section 11.1-11.5
\label{sec-1-1-3-1}
\textbf{Ordinary}: single variable
An \textbf{ODE} is \uline{separable} if it can be written in the form:
$f(y)\frac{dy}{dx}=g(x)$
Solve by "multiplying by dx" and integrating:
$\int{f(y)dy}=\int{g(x)dx}$

\textbf{First Order Linear DEs}
$y'+p(x)y=q(x)$
if $q(x)=0$, this is separable, and solvable using:
$y=e^{-P(x)}\int{e^{P(x)}q(x)dx}$, where $p'(x)=p(x)$

\begin{itemize}
\item $\square$ example of something similar the salt water one, or practise several of them
\end{itemize}
\item Functions of Several Variables, Section 12.1
\label{sec-1-1-3-2}
just graphs of shit, hopefully not important LUL
\end{enumerate}
\subsubsection{W11}
\label{sec-1-1-4}
\begin{enumerate}
\item Domains and Subsets of R$^{\text{n}}$, Section 12.2
\label{sec-1-1-4-1}
IDEK FUCK ME DADDY
\item Limits and Continuity, Section 13.1
\label{sec-1-1-4-2}
\item Partial Derivatives, Section 13.1
\label{sec-1-1-4-3}
$\frac{\delta f}{\delta x}=f_{1}=D_{1}f$
for $\frac{\delta f}{\delta x}$, differentiate for x, treat y as a constant, opposite for y
for the more complicated shit, $\frac{\delta^{2} f}{\delta x \delta y} = \frac{\delta f}{\delta y}(\frac{\delta f}{\delta x})$ and such like
Theorem of $\frac{\delta^{2} f}{\delta x \delta y}(a,b)=\frac{\delta^{2} f}{\delta y \delta x}(a,b)$. If this isn't working, check continuity of function
\item Linear Approximation, Section 13.2
\label{sec-1-1-4-4}
$f(x+ \Delta x, y+\Delta y) \approx f(x,y)+\frac{\delta f}{\delta x}(x,y)\Delta x + \frac{\delta f}{\delta y}(x,y)\Delta y$
the \emph{differential} of f(x,y) at (a,b) is
$df(a,b)=\frac{\delta f}{\delta x}(a,b)dx+\frac{\delta f}{\delta y}(a,b)dy$
\begin{itemize}
\item $\square$ example?
\end{itemize}
\item Differentiability, Section 13.3
\label{sec-1-1-4-5}
hand write this formula, fuck putting that in \LaTeX{}
f differentiable at (a,b) => f continuous at (a,b)
\item The Chain Rule, Section 13.4
\label{sec-1-1-4-6}
hand write this as well, it's gay to write but doesn't seem too hard to use, maybe an example but probs not as long as I've practiced some of them
\end{enumerate}
\subsubsection{W12}
\label{sec-1-1-5}
\begin{enumerate}
\item Gradients and Directional Derivatives, Section 13.5
\label{sec-1-1-5-1}
$\frac{\delta f}{\delta x}$ measures RoC in x-direction
$\frac{\delta f}{\delta y}$ measures RoC in y-direction
The \emph{gradient} of f is:
$\nabla f = \frac{\delta f}{\delta x}\vec{i} + \frac{\delta f}{\delta y}\vec{j} (+\frac{\delta f}{\delta z}\vec{k})$
looooot of shit for this one, make sure to practise
\item Extrema and Optimisation, Section 13.6
\label{sec-1-1-5-2}
Critical points:
\begin{description}
\item[{Stationary point}] where $\nabla f(a,b)=0$
\item[{Singular point}] where $\nabla f(a,b)$ does not exist
\item[{Boundary point}] where (a,b) is on the boundary of the domain of f
\end{description}
Local min/max can only occur at critical points
\texttt{there is a lot more than this, make sure to get the important parts}
\item Multivariable Integration, Sections 14.1-14.2
\label{sec-1-1-5-3}
\end{enumerate}
\subsubsection{W13}
\label{sec-1-1-6}
\begin{enumerate}
\item More Multivariable Integration, Sections 14.3-14.5
\label{sec-1-1-6-1}
\end{enumerate}
\subsection{Algebra}
\label{sec-1-2}
\subsubsection{W8}
\label{sec-1-2-1}
\begin{enumerate}
\item Complex Numbers, Lay App. B, Adams App. 1
\label{sec-1-2-1-1}
\item Invertibale Matrices, Lay 2.2-2.3
\label{sec-1-2-1-2}
\end{enumerate}
\subsubsection{W9}
\label{sec-1-2-2}
\begin{enumerate}
\item Determinants, Lay 3.1-3.2
\label{sec-1-2-2-1}
\end{enumerate}
\subsubsection{W10}
\label{sec-1-2-3}
\begin{enumerate}
\item More Determinants, Lay 3.2-3.3
\label{sec-1-2-3-1}
\end{enumerate}
\subsubsection{W11}
\label{sec-1-2-4}
\begin{enumerate}
\item Eigenvectors and Eigenvalues, Lay 5.1-5.2
\label{sec-1-2-4-1}
\item Diagonalisation, Lay 5.3
\label{sec-1-2-4-2}
\item Eigenvectors and Linear Transformations, Lay 5.4 (\emph{unfinished})
\label{sec-1-2-4-3}
\end{enumerate}
\subsubsection{W12}
\label{sec-1-2-5}
\begin{enumerate}
\item Eigenvectors and Linear Transformations, rest of Lay 5.4
\label{sec-1-2-5-1}
\item Applications to Differential Equations, Lay 5.7 (\emph{unfinished})
\label{sec-1-2-5-2}
\item Complex Eigenvalues, Lay 5.5 (\emph{unfinished})
\label{sec-1-2-5-3}
\end{enumerate}
\subsubsection{W13}
\label{sec-1-2-6}
\begin{enumerate}
\item More Complex Eigenvalues, Factoring as A=PCP$^{\text{-1}}$, rest of Lay 5.5
\label{sec-1-2-6-1}
\item Discrete and Continuous Dynamical System, Lay 5.6-5.7
\label{sec-1-2-6-2}
\end{enumerate}
\section{TODOstuff}
\label{sec-2}
\subsection{{\bfseries\sffamily TODO} Include examples noted in \^{}}
\label{sec-2-1}
\subsection{{\bfseries\sffamily TODO} Be sure of how to compute the various spaces and their bases of a matrix}
\label{sec-2-2}
% Emacs 25.1.1 (Org mode 8.2.10)
\end{document}